\section{Experimental}

\subsection{Benchmarks and verification} \label{sec:exp_benchmarks}

\subsubsection{Benchmarks for the brute force approach, no repulsion or Jastrow factor}
As discussed in section \ref{sec:motivation}, when $\omega = 1 \Hzn$, the trial wave function should be able to reproduce the exact solution $E=2 \Jn$ when we disregard the electron repulsion part of the Hamiltonian and don't include the Jastrow factor. 
This benchmark was tested\footnote{/Logs/N2\_norep\_bruteforce\_num/test\_investigate.cpp, 21.11.14. See appendix, section \ref{sec:codes}.}
with the brute force Metropolis method by varying $\alpha$ from $0$ to $1.5$ with steps of $0.05$ using numerical differentiation of the wave function in the expression of the local energy. 
$10^7$ Monte Carlo simulations were performed for each $\alpha$ with a step length $\Delta r$ suited to each case to get an acceptance rate of around $0.5$ (which is implemented in the code before the Monte Carlo simulation is started).

Then then the benchmarks for the $N=6$ and $N=12$ electron case was tested\footnote{/Logs/N12\_norep\_bruteforce\_num/test\_investigate.cpp and /Logs/N6\_norep\_bruteforce\_num/test\_investigate.cpp, 21.11.14. }
, still with the brute force approach and $10^7$ Monte Carlo simulations for the $N=6$ case and $10^6$ for the $N=12$ case, but this time with a smaller interval around $\alpha=1$, ranging from $0.9$ to $1.1$ with steps of $0.05$. 
In order to also verify the correct implementation of the oscillator frequency $\omega$, this was set to $1.5 \Hzn$, so the energies to reproduce are $10 \omega = 15 \Jn$ and $28 \omega = 42 \Jn$. 

\subsubsection{Benchmark for the brute force approach, with repulsion and Jastrow factor} \label{sec:exp_N2rep}

The exact energy of the two electron state \textit{with} repulsion has been shown \cite{Taut} to be $3 \omega$. 
To test this result, first a fast investigation of $\langle E_L\rangle$ was performed as function of $\alpha$ and $\beta$ to find the region in which the lowest energy is. 
Then, a more detailed search\footnote{/Logs/N2\_rep\_jast\_num/test\_investigate.cpp, 21.11.14.} was made with $\alpha \in [0.9,1.1]$ and $\beta \in [0.35,0.45]$, both in steps of $0.01$ and $10^6$ MC simulations at each step. 
The brute force approach with numerical evaluation of the local energy was used and $\omega$ was set to $1 \Hzn$.
If the exact wave function were within our trial parameters, we would expect to get the exact answer $3 \cdot 1 = 3 \Jn$, but since this may not be the case we expect the lowest energy to be larger than this, according to section \ref{sec:variational_principle}.

\subsubsection{Comparison of different methods} \label{sec:exp_methods_E}

As described in the theory section, a variety of different methods for solving the VMC problem has been explained. 
Firstly, there is a choice whether to use brute force (BF) or importance sampling (IS) when picking new trial positions in the metropolis algorithm.
Secondly there is the the possibility of using numerical methods (NLE) or the analytical expressions (ALE) when evaluating the local energy. 
In addition, if we're using importance sampling in the metropolis algorithm, there is a choice to whether or not we should 
use numerical (NQF) or analytical (AQF) expressions for the quantum force. 
All these methods should output the same result for the expectation value of the local energy, and to verify this an investigation\footnote{/Logs/compare\_methods/first\_example.cpp, second\_example.cpp, third\_example.cpp, 21.11.14.}
 of $\langle E_L \rangle$ with the different methods were performed for three different, semi-random\footnote{Chosen randomly by me, that is.}
, combinations of problem and trial function parameters; (Number of electrons $N$, $\alpha$, Jastrow Factor on (Jn) or off (Jf), $\beta$, $\omega$, Electron repulsion on (En) or off (Ef)) with $10^6$ Monte Carlo simulations.






















\subsection{Optimization and differences} \label{sec:exp_opt_and_diff}

In this section the different optimization methods, such as the Jastrow factor, analytical expressions and importance sampling, was investigated. 
The investigation of these issues was based upon twelve test cases with two electrons for which the optimal parameters were found, the procedure is explained in the following section.
Since the validity of the different methods were verified in section \ref{sec:exp_benchmarks}, the method which seemed most efficient was used. Therefore, if nothing else is written, the method used is the brute force method with analytical expressions for the local energy. 

\subsubsection{Test cases}\label{sec:exp_test_cases}

The twelve test cases for this section were based on 6 different oscillator strengths ($\omega =$ 0.01 0.1, 0.28, 0.5, 0.75, 1.0$\Hzn$) and electron repulsion both with and without the Jastrow factor.
For each test case, the optimal parameters of $\alpha$ and $\beta$ was found by calculating the energies for $\alpha$ and $\beta$ in range $[0,1.2]$ with a resolution of $0.01$ and $10^5$ MC simulations.
Then the parameters yielding the smallest energy was logged\footnote{/Logs/find\_parameters2/, 22.11.14.}. 

\subsubsection{Jastrow factor}

When the repulsion part is present, we expect the Jastrow factor to improve the correlation (interaction) between the electrons, giving us a better (i.e. lower) estimate of the energy. 
We can measure the correlation by seeing how the lowest energy estimate changes with the Jastrow factor relative to without the Jastrow factor. 
This relative difference was calculated for each of the test cases. 

\subsubsection{Importance sampling}

Using importance sampling introduces the dependence on $\delta t$ when evaluating $\langle E_L \rangle$, as discussed in section \ref{sec:theory_importance_sampling}.
The dependence of the importance sampling method on $\delta t$ was investigated\footnote{Logs/importance\_sampling/, 24.11.14.} 
for $10^3 - 10^6$ Monte Carlo simulations and $\delta t \in [10^{-6}, 10^2]$. 
Trial wave function parameters were chosen to be $\alpha = 0.72$ and $\beta = 0.24$\footnote{Which is not close to an eigenstate of the system for which we found optimal parameters in section \ref{sec:res_N2_rep}. This is important because if the parameters are such that we are close to an eigenstate, $E_L$ will not vary much from point to point, which reduces the importance of good sampling.}.
For every time step $\delta t$, a brute force evaluation of the energy was also performed for comparison. 
The acceptance rate as a function of $\delta t$ was also investigated\footnote{/Logs/IS\_accept/IS\_accept.cpp, 07.12.14.} with $10^7$ Monte Carlo simulations.


\subsubsection{Timely differences between methods}

The time used by the different methods was investigated for $N=2$\footnote{/Logs/times1/, 25.11.14.}  and 
$N=6$\footnote{/Logs/times2/, 25.11.14} electrons with some arbitrarily chosen values of $\alpha$ and $\beta$. 
The notation of the methods is described in section \ref{sec:exp_methods_E}. 


















\subsection{Applications}

\subsubsection{Properties of the approximated wave functions}


Another six test cases and wave functions with $6$ electrons were found\footnote{/Logs/find\_parameters2/, 22.11.14.}  using the same methods described in section \ref{sec:exp_test_cases}\footnote{For the $\omega=0.01$ case, a manual search, as the one in section \ref{sec:exp_N2rep}, had to be made when the result from the automated search yielded negative energies. /Logs/N6w001/, 28.11.14.}.
Then five interesting properties of the 24 approximated wave functions found here and in section \ref{sec:res_test_cases} was investigated\footnote{Logs/interesting\_quantities, 29.11.14.} with $10^7$ Monte Carlo simulations.
These five interesting properties were

\begin{itemize}
	\item The expectation value of the energy
	\item The variance of the local energy
	\item The expectation value of the average distance between electrons
	\item The expectation value of the potential energy of the system 
	\item The expectation value of the kinetic energy of the system
\end{itemize}

\subsubsection{The virial theorem }

The virial theorem, as described in section \ref{sec:theory_virial_theorem}, was tested for the optimal wavefunctions found in these earlier sections by plotting plotting the ratio of $\langle T \rangle$ to $\langle V \rangle$ as a function of $\omega$.
 The different forms of energy were also found\footnote{/Logs/interesting\_quantities2, 29.11.14.} for the no repulsion, no Jastrow factor case, for which the optimal parameter is $\alpha = 1$, as found in section \ref{sec:res_N2N6_norep}, in order to test the virial theorem for these cases too. 
