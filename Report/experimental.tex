\section{Experimental}

\subsection{Benchmarks and verification} \label{sec:exp_benchmarks}

\subsubsection{Benchmarks for the brute force approach, no repulsion or jastrow factor}
As discussed in section \ref{sec:motivation}, when $\omega = 1$, the trial wavefunction should be able to reproduce the exact solution $E=2$ (in atomic units) when we disregard the electron repulsion part of the Hamiltonian and don't include the Jastrow factor. 
This benchmark was tested
\footnote{/Logs/N2\_norep\_bruteforce\_num/test\_investigate.cpp, 21.11.14. See appendix, section \ref{sec:codes}.}
with the brute force metropolis method by varying $\alpha$ from $0$ to $1.5$ with steps of $0.05$ using numerical diffrentiation of the wavefunction in the expression of the local energy. 
$10^7$ Monte Carlo simulations were performed for each $\alpha$ with a step length $\Delta r$ suited to each case to get an acceptance rate of around $0.5$ (which is implemented in the code before any monte carlo simulation is begun).

Then then the benchmarks for the $N=6$ and $N=12$ electron case was tested\footnote{/Logs/N12\_norep\_bruteforce\_num/test\_investigate.cpp and /Logs/N6\_norep\_bruteforce\_num/test\_investigate.cpp, 21.11.14. }
, still with the brute force approach and $10^7$ monte carlo simulations for the $N=6$ case and $10^6$ for the $N=12$ case, but this time with a smaller interval around $\alpha=1$, ranging from $0.9$ to $1.1$ with steps of $0.05$. 
In order to also verify the correct implementation of the oscillator frequency $\omega$, this was set to $1.5$, so the energies to reproduce are $10 \omega = 15$ a.u. and $28 \omega = 42$ a.u. 

\subsubsection{Benchmark for the brute force approach, with repulsion and jastrow factor}

The exact energy of the two electron state \textit{with} repulsion has been shown \cite{Taut} to be $3 \omega$. 
To test this result, first a fast investigation of $\langle E_L\rangle$ was performed as function of $\alpha$ and $\beta$ to find the region in which the lowest energy is. 
Then, a more detailed search\footnote{/Logs/N2\_rep\_jast\_num/test\_investigate.cpp, 21.11.14.} was made with $\alpha \in [0.9,1.1]$ and $\beta \in [0.35,0.45]$, both in steps of $0.01$ and $10^6$ MC simulations at each step. 
The brute force approach with numerical evaluation of the local energy was used and $\omega$ was set to $1$.
If the exact wavefunction were within our trial parameters, then we would thus expect to get the exact answer $3 \cdot 1 = 3$ a.u., but since this may not be the case we expect the lowest energy to be larger than this, according to section \ref{sec:variational_principle}.

\subsubsection{Comparison of different methods} \label{sec:exp_methods_E}

As described in the theory section, a variety of different methods for solving the VMC problem has been explained. 
Firstly, there is a choice whether to use brute force (BF) or importance sampling (IS) when picking new trial positions in the metropolis algorithm.
Secondly there is the the possibility of using numerical methods (NLE) or the analytical expressions (ALE) when evaluating the local energy. 
In addition, if we're using importance sampling in the metropolis algorithm, there is a choice to whether or not we should 
use numerical (NQF) or analytical (AQF) expressions for the quantum force. 
All these methods should output the same result for the expectation value of the local energy, and to verify this an investigation\footnote{/Logs/compare\_methods/first\_example.cpp, second\_example.cpp, third\_example.cpp, 21.11.14.}
 of $\langle E_L \rangle$ with the different methods were performed for three different, semi-random\footnote{Chosen randomly by me, that is.}
, combination of problem and trial function parameters; (Number of electrons $N$, $\alpha$, Jastrow Factor on (Jn) or off (Jf), $\beta$, $\omega$, Electron repulsion on (En) or off (Ef)) with $10^6$ monte carlo simulations.






















\subsection{Optimizations and differences} \label{sec:exp_opt_and_diff}

In this section, the different optimization methods, such as the jastrow factor, analytical expressions and importance sampling, was investigated. 
The investigation of these issues was based upon six test cases with two electrons for which the optimal parameters were found, the procedure is explained in the following section.
Since we have verified the validity of the different methods in section \ref{sec:exp_benchmarks}, we could use whatever method we wanted to run sample the energies. 
For this section, if nothing else is written, the standard method is the brute force approach with analytical expressions for the local energy. 

\subsubsection{Test cases}

The six test cases for this section was $\omega = 0.01, 0.28, 1.0$, with repulsion but with and without Jastrow factor.
For each test case, the optimal parameters of $\alpha$ and $\beta$ was found by first doing a high resolution (step: 0.2) sweep and after finding a region of interest, incresing the resolution (step: 0.01) over a smaller area. 
Then the parameters yielding the smallest energy was logged. 
The configuration files for each run was logged \footnote{/Logs/test\_cases/, 22.11.14}. 

\subsubsection{Jastrow factor}

When the repulsion part is present, we expect the jastrow factor to 

\subsubsection{Importance sampling}

\subsubsection{Timely differences between methods}



















\subsection{Applications}

\subsubsection{Energies and variances}

\subsubsection{The virial theorem }